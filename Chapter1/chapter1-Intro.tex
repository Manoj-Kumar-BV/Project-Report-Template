\chapter{Introduction to Analog and Digital converters}

\section[Introduction]{\textbf{Introduction}}
The title of the project can be introduced in this section. This section should neatly elaborate the context of the project, the relevance of the area chosen and the title. You can bring a brief history and arrive at the title of the project. Use appropriate number of paragraphs within this section. 

You are allowed to use figures or diagrams which can help in introducing the topic acknowledging the source. For example , if you are introducing a particular topic, an appropriate figure can be used. The figure should be referenced in the text as Figure. \ref{fig:universe} 
\begin{figure}[htb]
\centering
	\includegraphics[scale=1]{Figures/universe}	
	\caption{Sample picture of universe }
	\label{fig:universe}
\end{figure}

These guidelines are provided to formally expose you to the various ethical and technical issues involved in writing up your work and the format you are required to adhere to while submitting your project report.

\section[Literature Review]{\textbf{Literature Review}}

A literature review is a text of a scholarly paper, which includes the current knowledge including substantive findings, as well as theoretical and methodological contributions to a particular topic. Literature reviews are secondary sources, and do not report new or original experimental work. Most often associated with academic-oriented literature, such reviews are found in academic journals, and are not to be confused with book reviews that may also appear in the same publication. Literature reviews are a basis for research in nearly every academic field . A narrow-scope literature review may be included as part of a peer-reviewed journal article presenting new research, serving to situate the current study within the body of the relevant literature and to provide context for the reader. In such a case, the review usually precedes the methodology and results sections of the work.

\subsection{Sample}
The main types of literature reviews are: evaluative, exploratory, and instrumental. A fourth type, the systematic review, is often classified separately, but is essentially a literature review focused on a research question, trying to identify, appraise, select and synthesize all high-quality research evidence and arguments relevant to that question. A meta-analysis is typically a systematic review using statistical methods to effectively combine the data used on all selected studies to produce a more reliable result.
\subsubsection[Review types]{\textbf{Review types}}

The main types of literature reviews are: evaluative, exploratory, and instrumental. A fourth type, the systematic review, is often classified separately, but is essentially a literature review focused on a research question, trying to identify, appraise, select and synthesize all high-quality research evidence and arguments relevant to that question. A meta-analysis is typically a systematic review using statistical methods to effectively combine the data used on all selected studies to produce a more reliable result.


\subsubsection[Process and product]{\textbf{Process and product}}

Distinguish between the process of reviewing the literature and a finished work or product known as a literature review. The process of reviewing the literature is often ongoing and informs many aspects of the empirical research project. All of the latest literature should inform a research project. Scholars need to be scanning the literature long after a formal literature review product appears to be completed.

\subsubsection{\textbf{Page limitation}}

A careful literature review is usually 15 to 30 pages and could be longer. The process of reviewing the literature requires different kinds of activities and ways of thinking and link the activities of doing a literature review with Benjamin Bloom’s revised taxonomy of the cognitive domain (ways of thinking: remembering, understanding, applying, analysing, evaluating, and creating).

This section should contain the review of the literature in the past.You should review a minimum of 10 papers from standard reference journals. Kindly avoid local conference papers and papers from predatory journals. Kindly consult with your guide and finalize papers to be considered for review before adding in this section.Report the major observations and findings from each paper in one paragraph in the format given below.

proposed various techniques for adders and multipliers.Add the reference papers to the bibliography section using Jabref and cite it here using the instructions given in further chapters.


\subsubsection{\textbf{Plagiarism}}

To use someone else's exact words without quotation marks and appropriate credit, or to use the unique ideas of someone else without acknowledgement, is known as plagiarism. In publishing, plagiarism is illegal; in other circumstances, it is, at the least, unethical. You may quote or paraphrase the words or ideas of another if you document your source. Although you need not enclose the paraphrased material in quotation marks, you must document the source. 

Paraphrased ideas are taken from someone else whether or not the words are identical. Paraphrasing a passage without citing the source is permissible only when the information paraphrased is common knowledge in a field. (Common knowledge refers to historical, scientific, geographical, technical, and other type of information on a topic readily available in handbooks, manuals, atlases and other references). 

\subsubsection{How to add Reference}
Use \texttt{Jabref} which will help in adding the reference in a separate file, from which one can use \verb|\citep\{\}| command to add reference. A sample, referring to a textbook would look something like this,\cite{Razavi2000}.

\section[Motivation]{\textbf{Motivation}}

Brief the motivation of selecting your project title. You can elaborate the challenges in the specific area, relevance and importance of the chosen topic. 

\section[Problem statement]{\textbf{Problem statement}}

Define the problem statement in this section, in one paragraph.

\section[Objectives]{\textbf{Objectives}}
The objectives of the project are
\begin{enumerate}
\item To design a pipelined ADC for audio frequency range
\item List all the objectives in the above format , starting with "To"
\item Limit the number of objectives to a maximum of three
\end{enumerate}

\section[Brief Methodology of the project]{\textbf{Brief Methodology of the project}}
Discuss about the methodology you identified to execute the objectives of your project in brief. Methodology is a system of practices, techniques, procedures, and rules used to execute a particular project. You can elaborate the methodology in a later chapter. Here you can present in the form of a flow diagram and explain the methodology in a paragraph.

\section[Assumptions made / Constraints of the project]{\textbf{Assumptions made / Constraints of the project}}

List the assumptions made for the execution of the project in this section. You can also elaborate on the major constraints of the project. This section should clearly state under what conditions your project is valid. It is mandatory to have this section in your project report.

\section[Organization of the report]{\textbf{Organization of the report}}

This report is organized as follows. Write the discussions in each chapter. A sample is as follows.
\begin{itemize}
\item Chapter 2 discusses the fundamentals of ADC and the performance parameters for evaluation.
\item Chapter 3 discusses .
\item Chapter 4 discusses .
\item Chapter 5 discusses .
\item Chapter 6 discusses .
\end{itemize}

.